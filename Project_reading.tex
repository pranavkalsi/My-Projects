% Created 2025-05-10 Sat 18:22
% Intended LaTeX compiler: pdflatex
\documentclass[11pt]{article}
\usepackage[utf8]{inputenc}
\usepackage[T1]{fontenc}
\usepackage{graphicx}
\usepackage{longtable}
\usepackage{wrapfig}
\usepackage{rotating}
\usepackage[normalem]{ulem}
\usepackage{amsmath}
\usepackage{amssymb}
\usepackage{capt-of}
\usepackage{hyperref}
\author{Pranav Kalsi}
\date{\today}
\title{Project Reading}
\hypersetup{
 pdfauthor={Pranav Kalsi},
 pdftitle={Project Reading},
 pdfkeywords={},
 pdfsubject={},
 pdfcreator={Emacs 30.1 (Org mode 9.6.7)}, 
 pdflang={English}}
\usepackage{biblatex}
\addbibresource{/home/ega-ninja/Projects/Gaia_converging point/biblio.bib}
\addbibresource{~/exocortex/bibliography.bib}
\begin{document}

\maketitle
\tableofcontents

This file contains the important and relevant things from the reading resources for the project. It also contains doubts which are unresolved.
Last edited: \textit{[2025-05-10 Sat] } 
\section{Converging point method}
\label{sec:orged6e051}
The method is beautifully explained in this website: \url{https://physics.weber.edu/palen/clearinghouse/labs/hyades/disthyad.html}.

\section{Reading Paper:}
\label{sec:org08af745}

\subsection{4. Membership determination:}
\label{sec:org21eb5b5}
Using Gaia DR2, data proper-motion is plotted as Vector Point Diagram (VPD). And the cluster members are assigned by chosing   stars which form a dense cluster of points in VPD.
\begin{quote}
    The centre of the circular region confining the
probable cluster members was determined by maximum density
method in the proper motion plane which is found to lie at (μx , μy )
≡ (μα cosθ, μδ ) ≈ (−0.13, −3.37) mas yr−1 .
The radius of the circle
was derived by plotting the stellar density as a function of radial
distance in the proper motion plane as illustrated in Fig. 2. We fit the stellar density profile with a function similar to the one used to characterize the radial profiles of star clusters in the galaxies.
\end{quote}
\subsubsection{Membership probability}
\label{sec:org7183102}
\begin{quote}
\[
P_\mu(i) = \frac{n_c \cdot \phi^v_c(i)}{n_c \cdot \phi^v_c(i) + n_f \cdot \phi^v_f(i)}
\]
\[
\phi^v_c(i) = \frac{1}{2\pi \sqrt{(\sigma_{xc}^2 + \epsilon_{xi}^2)(\sigma_{yc}^2 + \epsilon_{yi}^2)}} \exp\left\{ - \frac{1}{2} \left[ \frac{(\mu_{xi} - \mu_{xc})^2}{\sigma_{xc}^2 + \epsilon_{xi}^2} + \frac{(\mu_{yi} - \mu_{yc})^2}{\sigma_{yc}^2 + \epsilon_{yi}^2} \right] \right\}
\]

\[
\phi^v_f(i) = \frac{1}{2\pi \sqrt{1 - \gamma^2} \sqrt{(\sigma_{xf}^2 + \epsilon_{xi}^2)(\sigma_{yf}^2 + \epsilon_{yi}^2)}}
\
\exp\left\{ - \frac{1}{2(1 - \gamma^2)} \left[ \frac{(\mu_{xi} - \mu_{xf})^2}{\sigma_{xf}^2 + \epsilon_{xi}^2} - \frac{2\gamma (\mu_{xi} - \mu_{xf})(\mu_{yi} - \mu_{yf})}{\sqrt{(\sigma_{xf}^2 + \epsilon_{xi}^2)(\sigma_{yf}^2 + \epsilon_{yi}^2)}} - \frac{(\mu_{yi} - \mu_{yf})^2}{\sigma_{yf}^2 + \epsilon_{yi}^2} \right] \right\}
\]


\[
\gamma = \frac{(\mu_{xi} - \mu_{xf})(\mu_{yi} - \mu_{yf})}{\sigma_{xf} \sigma_{yf}}
\]
\end{quote}
\end{document}